\rhead{\bf mc\_SetDefaultAgentStatus()}
\noindent
\vspace{5pt}
\rule{6.5in}{0.015in}
\noindent
{\LARGE \bf mc\_SetDefaultAgentStatus()\index{mc\_SetDefaultAgentStatus()}}\\
\phantomsection
\addcontentsline{toc}{section}{mc\_SetDefaultAgentStatus()}

\noindent
{\bf Synopsis}\\
%{\bf \#include $<$mobilec.h$>$}\\
{\bf int mc\_SetDefaultAgentStatus}({\bf int} $status$);\\

\noindent
{\bf Purpose}\\
Set the default status of any incoming mobile agents.\\

\noindent
{\bf Return Value}\\
This function returns 0 on success and non-zero otherwise.

\noindent
{\bf Parameters}
\vspace{-0.1in}
\begin{description}
\item
\begin{tabular}{p{10 mm}p{145 mm}}
$status$ & An integer representing the status to be assinged to any incoming
mobile agents as their default status.
\end{tabular}
\end{description}

\noindent
{\bf Description}\\
This function sets the default status of any incoming mobile agents by one of 
the enumerated values of type {\texttt enum mc\_AgentStatus\_e}. See Table
\vref{tab:agent_status_enums} for a complete listing of the enumerated type.

\noindent
{\bf Example}\\
Please see the example for MC\_SetDefaultAgentStatus() on page
\pageref{api:MC_SetDefaultAgentStatus()}.
\noindent
%Compare with output for examples in \CPlot::\Arrow(), \CPlot::\AutoScale(),
%\CPlot::\DisplayTime(), \CPlot::\Label(), \CPlot::\TicsLabel(), 
%\CPlot::\Margins(), \CPlot::\BoundingBoxOffsets(), \CPlot::\TicsDirection(),\linebreak
%\CPlot::\TicsFormat(), and \CPlot::\Title().
%{\footnotesize\verbatiminput{template/example/Data2D.ch}}

\noindent
{\bf See Also}\\

%\CPlot::\DataThreeD(), \CPlot::\DataFile(), \CPlot::\Plotting(), \plotxy().\\
