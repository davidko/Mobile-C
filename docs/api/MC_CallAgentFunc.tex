\rhead{\bf MC\_CallAgentFunc()}
\noindent
\vspace{5pt}
\rule{6.5in}{0.015in}
\noindent
\phantomsection
{\LARGE \bf MC\_CallAgentFunc()\index{MC\_CallAgentFunc()}}\\
\addcontentsline{toc}{section}{MC\_CallAgentFunc()}

\noindent
{\bf Synopsis}\\
{\bf \#include $<$libmc.h$>$}\\
{\bf int MC\_CallAgentFunc}({\bf MCAgent\_t} $agent$, {\bf const char*} $funcName$, {\bf void*} $returnVal$, ...);\\

\noindent
{\bf Purpose}\\
This function is used to call a function that is defined in an agent. \\

\noindent
{\bf Return Value}\\
This function returns 0 on success, or a non-zero error code on failure. \\

\noindent
{\bf Parameters}
\vspace{-0.1in}
\begin{description}
\item
\begin{tabular}{ll}
$agent$ & The agent in which to call a function. \\
$funcName$ & The function to call. \\
$returnVal$ & (Output) The return value of the agent function. \\
$...$ &  Arguments to pass to the function. 
\end{tabular}
\end{description}

\noindent
{\bf Description}\\
This function enables a program to treat agents as libraries of functions. Thus, an agent
may provide a library of functions that may be called from binary space with this function, 
or from another agent by the agent-space version of this function. \\

\noindent
{\bf Example}\\
\noindent
{\footnotesize\verbatiminput{../demos/cspace-agentspace_interface/persistent_example/server.c}}

\noindent
{\bf See Also}\\
MC\_CallAgentFuncVar()

%\CPlot::\DataThreeD(), \CPlot::\DataFile(), \CPlot::\Plotting(), \plotxy().\\
