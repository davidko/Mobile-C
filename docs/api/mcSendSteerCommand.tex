\rhead{\bf mc\_SendSteerCommand()}
\noindent
\vspace{5pt}
\rule{6.5in}{0.015in}
\noindent
\phantomsection
{\LARGE \bf mc\_SendSteerCommand()\index{mc\_SendSteerCommand()}}\\
\addcontentsline{toc}{section}{mc\_SendSteerCommand()}

\noindent
{\bf Synopsis}\\
{\bf \#include $<$libmc.h$>$}\\
{\bf int mc\_SendSteerCommand}({\bf MCAgency\_t} attr, {\bf int(*)(void* data)} funcptr, {\bf void*} arg);\\

\noindent
{\bf Purpose}\\
The mc\_SendSteerCommand function sends a computational steering command to the
algorithm at the agent's current agency.

\noindent
{\bf Return Value}\\
The function returns 0 on success, or a non-zero error code on failure. \\

\noindent
{\bf Description}\\
This function enables mobile agents to send steer commands to steering-enables
algorithms running at the agent's local agency. See the demo at 
demos/steer\_example/ for more details. \\

{\bf Example}\\
\noindent
{\footnotesize \verbatiminput{../demos/miscellaneous/steer_example/resume.xml}}
%Compare with output for examples in \CPlot::\Arrow(), \CPlot::\AutoScale(),
%\CPlot::\DisplayTime(), \CPlot::\Label(), \CPlot::\TicsLabel(), 
%\CPlot::\Margins(), \CPlot::\BoundingBoxOffsets(), \CPlot::\TicsDirection(),\linebreak
%\CPlot::\TicsFormat(), and \CPlot::\Title().
%{\footnotesize\verbatiminput{template/example/Data2D.ch}}

\noindent
{\bf See Also}\\
MC\_Steer(), MC\_SteerControl()

%\CPlot::\DataThreeD(), \CPlot::\DataFile(), \CPlot::\Plotting(), \plotxy().\\
