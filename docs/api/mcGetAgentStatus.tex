\rhead{\bf mc\_GetAgentStatus()}
\noindent
\vspace{5pt}
\rule{6.5in}{0.015in}
\noindent
{\LARGE \bf mc\_GetAgentStatus()\index{mc\_GetAgentStatus()}}\\
\phantomsection
\addcontentsline{toc}{section}{mc\_GetAgentStatus()}

\noindent
{\bf Synopsis}\\
{\bf \#include $<$mobilec.h$>$}\\
{\bf int mc\_GetAgentStatus}({\bf MCAgent\_t} $agent$);\\

\noindent
{\bf Purpose}\\
Get the status of a mobile agent in an agency.\\

\noindent
{\bf Return Value}\\
This function returns an enumerated value representing the current
status of a mobile agent. See Table \vref{tab:agent_status_enums}.\\

\noindent
{\bf Parameters}
\vspace{-0.1in}
\begin{description}
\item
\begin{tabular}{p{10 mm}p{145 mm}}
$agent$ & The mobile agent from which to retrieve status information.
\end{tabular}
\end{description}

\noindent
{\bf Description}\\
This function gets a mobile agent's status. 
The status is used to determine the mobile agent's current state of execution.\\

\noindent
{\bf Example}\\
This function is identical to the binary space version, MC\_GetAgentStatus().
Please see the documentation for MC\_GetAgentStatus on page 
\pageref{api:MC_GetAgentStatus} for an example.
\noindent
%Compare with output for examples in \CPlot::\Arrow(), \CPlot::\AutoScale(),
%\CPlot::\DisplayTime(), \CPlot::\Label(), \CPlot::\TicsLabel(), 
%\CPlot::\Margins(), \CPlot::\BoundingBoxOffsets(), \CPlot::\TicsDirection(),\linebreak
%\CPlot::\TicsFormat(), and \CPlot::\Title().
%{\footnotesize\verbatiminput{template/example/Data2D.ch}}

\noindent
{\bf See Also}\\

%\CPlot::\DataThreeD(), \CPlot::\DataFile(), \CPlot::\Plotting(), \plotxy().\\
