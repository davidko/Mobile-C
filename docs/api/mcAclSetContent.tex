\rhead{\bf mc\_AclSetContent()}
\noindent
\vspace{5pt}
\rule{6.5in}{0.015in}
\noindent
\phantomsection
{\LARGE \bf mc\_AclSetContent()\index{mc\_AclSetContent()}}\\
\addcontentsline{toc}{section}{mc\_AclSetContent()}
\label{api:mc_AclSetContent()}

\noindent
{\bf Synopsis}\\
{\bf \#include $<$libmc.h$>$}\\
{\bf int mc\_AclSetContent}({\bf fipa\_acl\_message\_t*} acl, {\bf const char*} name);\\

\noindent
{\bf Purpose}\\
Set the content on an ACL message.\\

\noindent
{\bf Return Value}\\
Returns 0 on success or non-zero on failure.\\

\noindent
{\bf Parameters}
\vspace{-0.1in}
\begin{description}
\item
\begin{tabular}{p{10 mm}p{145 mm}} 
$acl$ & An initialized ACL message. \\
$content$ & Set the content field of an ACL message.
\end{tabular}
\end{description}

\noindent
{\bf Description}\\
This function sets the ``content'' field of an ACL message. \\

\noindent
{\bf Example}\\
\noindent
{\footnotesize \verbatiminput{../demos/FIPA_compliant_ACL_messages/fipa_test/test2.xml}}

\noindent
{\bf See Also}\\
\texttt{
  mc\_AclSetPerformative(), mc\_AclSetSender(), mc\_AclAddReceiver(), 
    \linebreak mc\_AclAddReplyTo()
}

%\CPlot::\DataThreeD(), \CPlot::\DataFile(), \CPlot::\Plotting(), \plotxy().\\
