\rhead{\bf mc\_CondReset()}
\noindent
\vspace{5pt}
\rule{6.5in}{0.015in}
\noindent
\phantomsection
{\LARGE \bf mc\_CondReset()\index{mc\_CondReset()}}\\
\addcontentsline{toc}{section}{mc\_CondReset()}

\noindent
{\bf Synopsis}\\
%{\bf \#include $<$mobilec.h$>$}\\
{\bf int mc\_CondReset}({\bf int} $id$);\\

\noindent
{\bf Purpose}\\
Reset a Mobile-C Condition variable for re-use.\\

\noindent
{\bf Return Value}\\
This function returns 0 upon success or non-zero if the condition 
variable was not found. \\

\noindent
{\bf Parameters}
\vspace{-0.1pt}
\begin{description}
\item
\begin{tabular}{p{10 mm}p{145 mm}}
$id$ & The id of the condition variable to signal.
\end{tabular} 
\end{description}

\noindent
{\bf Description}\\
This function resets a used condition variable, setting it's state
back to an unsignalled state. A Mobile-C condition variable will
remain in a signalled state indefinitely until this function is called.\\

\noindent
{\bf Example}\\
See Program \vref{prog:cond_agent_1} and Program
\vref{prog:cond_agent_2} in Chapter \ref{chap:synchronization}.\\
\noindent

\noindent
{\bf See Also}\\
mc\_CondDelete(), mc\_CondInit(), mc\_CondSignal(), mc\_CondWait().\\

%\CPlot::\DataThreeD(), \CPlot::\DataFile(), \CPlot::\Plotting(), \plotxy().\\
