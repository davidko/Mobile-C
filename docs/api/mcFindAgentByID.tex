\rhead{\bf mc\_FindAgentByID()}
\noindent
\vspace{5pt}
\rule{6.5in}{0.015in}
\noindent
{\LARGE \bf mc\_FindAgentByID()\index{mc\_FindAgentByID()}}\\
\phantomsection
\addcontentsline{toc}{section}{mc\_FindAgentByID()}
\label{api:mc_FindAgentByID()}

\noindent
{\bf Synopsis}\\
%{\bf \#include $<$mobilec.h$>$}\\
{\bf MCAgent\_t MC\_FindAgentByID}({\bf int} $id$);\\

\noindent
{\bf Purpose}\\
Find a mobile agent by its ID number in a given agency.\\

\noindent
{\bf Return Value}\\
The function returns an {\bf MCAgent\_t} object on success or NULL on failure.\\

\noindent
{\bf Parameters}
\vspace{-0.1in}
\begin{description}
\item               
\begin{tabular}{p{10 mm}p{145 mm}}
$id$ & An integer representing a mobile agent's ID number.
\end{tabular}
\end{description}

\noindent
{\bf Description}\\
This function is used to find and retrieve a pointer to an existing running 
mobile agent in an agency by the mobile agent's ID number.\\

\noindent
{\bf Example}\\
\noindent
{\footnotesize\verbatiminput{../demos/agent_space_functionality/mc_df_service_test/test1.xml}}

\noindent
{\bf See Also}\\

%\CPlot::\DataThreeD(), \CPlot::\DataFile(), \CPlot::\Plotting(), \plotxy().\\
