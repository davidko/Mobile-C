\rhead{\bf mc\_AclPost()}
\noindent
\vspace{5pt}
\rule{6.5in}{0.015in}
\noindent
\phantomsection
{\LARGE \bf mc\_AclPost()\index{mc\_AclNew()}}\\
\addcontentsline{toc}{section}{mc\_AclPost()}
\label{api:mc_AclPost()}

\noindent
{\bf Synopsis}\\
{\bf \#include $<$libmc.h$>$}\\
{\bf int mc\_AclPost}({\bf mcAgent\_t} agent, {\bf fipa\_acl\_message\_t*} message);\\

\noindent
{\bf Purpose}\\
Post a message directly to an agent's mailbox.\\

\noindent
{\bf Return Value}\\
Returns 0 on success, non-zero on failure.\\

\noindent
{\bf Parameters}
\vspace{-0.1in}
\begin{description}
\item
\begin{tabular}{p{10 mm}p{145 mm}} 
$agent$ & An initialized mobile agent. \\
$message$ & The ACL message to post. 
\end{tabular}
\end{description}

\noindent
{\bf Description}\\
This function is used to post an ACL message directly to an agent's 
mailbox. The agent must reside on the same agency as the caller.
No forwarding or checking of any fields of the ACL message is performed.\\

\noindent
{\bf Example}\\
\noindent

\noindent
{\bf See Also}\\
\texttt{
  mc\_AclNew(), mc\_AclReply(), mc\_AclRetrieve(), mc\_AclSend(), 
    \linebreak mc\_AclWaitRetrieve()
}

%\CPlot::\DataThreeD(), \CPlot::\DataFile(), \CPlot::\Plotting(), \plotxy().\\
