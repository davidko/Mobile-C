\rhead{\bf mc\_AclSetPerformative()}
\noindent
\vspace{5pt}
\rule{6.5in}{0.015in}
\noindent
\phantomsection
{\LARGE \bf mc\_AclSetPerformative()\index{mc\_AclSetPerformative()}}\\
\addcontentsline{toc}{section}{mc\_AclSetPerformative()}
\label{api:mc_AclSetPerformative()}

\noindent
{\bf Synopsis}\\
{\bf \#include $<$libmc.h$>$}\\
{\bf int mc\_AclSetPerformative}({\bf fipa\_acl\_message\_t*} acl, enum fipa\_performative\_e performative);\\

\noindent
{\bf Purpose}\\
Set the performative on an ACL message.\\

\noindent
{\bf Return Value}\\
Returns 0 on success or non-zero on failure.

\noindent
{\bf Parameters}
\vspace{-0.1in}
\begin{description}
\item
\begin{tabular}{p{10 mm}p{145 mm}} 
$acl$ & An initialized ACL message. \\
$performative$ & The FIPA performative you wish the message to contain.
\end{tabular}
\end{description}

\noindent
{\bf Description}\\
This function is used to set the FIPA ACL performative on an ACL message. 
The performative may be any valid FIPA performative listed in the table 
below. \\
\begin{tabular}{ll}
Enumerated Value & FIPA Perfomative \\ \hline
\texttt{FIPA\_ACCEPT\_PROPOSAL} & accept-proposal \\
\texttt{FIPA\_AGREE} & agree \\
\texttt{FIPA\_CANCEL} & cancel\\
\texttt{FIPA\_CALL\_FOR\_PROPOSAL} & call-for-proposal \\
\texttt{FIPA\_CONFIRM} & confirm\\
\texttt{FIPA\_DISCONFIRM} & disconfirm \\
\texttt{FIPA\_FAILURE} & failure \\
\texttt{FIPA\_INFORM} & inform \\
\texttt{FIPA\_INFORM\_IF} & inform-if \\
\texttt{FIPA\_INFORM\_REF} & inform-ref \\
\texttt{FIPA\_NOT\_UNDERSTOOD} & not-understood \\
\texttt{FIPA\_PROPOGATE} & propogate \\
\texttt{FIPA\_PROPOSE} & propose \\
\texttt{FIPA\_PROXY} & proxy \\
\texttt{FIPA\_QUERY\_IF} & query-if \\
\texttt{FIPA\_QUERY\_REF} & query-ref \\
\texttt{FIPA\_REFUSE} & refuse \\
\texttt{FIPA\_REJECT\_PROPOSAL} & reject-proposal \\
\texttt{FIPA\_REQUEST} & request \\
\texttt{FIPA\_REQUEST\_WHEN} & request-when \\
\texttt{FIPA\_REQUEST\_WHENEVER} & request-whenever \\
\texttt{FIPA\_SUBSCRIBE} & subscribe
\end{tabular}


\noindent
{\bf Example}\\
\noindent
{\footnotesize\verbatiminput{../demos/FIPA_compliant_ACL_messages/fipa_test/test2.xml}}

\noindent
{\bf See Also}\\
\texttt{
  mc\_AclSetSender(), mc\_AclAddReceiver(), mc\_AclAddReplyTo(), 
    \linebreak mc\_AclSetContent()
}

%\CPlot::\DataThreeD(), \CPlot::\DataFile(), \CPlot::\Plotting(), \plotxy().\\
