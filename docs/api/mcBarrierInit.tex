\rhead{\bf mc\_BarrierInit()}
\noindent
\vspace{5pt}
\rule{6.5in}{0.015in}
\noindent
\phantomsection
{\LARGE \bf mc\_BarrierInit()\index{mc\_BarrierInit()}}\\
\addcontentsline{toc}{section}{mc\_BarrierInit()}
\label{api:mc_BarrierInit()}

\noindent
{\bf Synopsis}\\
{\bf int mc\_BarrierInit}({\bf int} $id$, {\bf int} $num\_procs$);\\

\noindent
{\bf Purpose}\\
This function initializes a Mobile-C Barrier variable for usage. \\

\noindent
{\bf Return Value}\\
This function returns 0 on success, or non-zero if the id could not be found. \\

\noindent
{\bf Parameters}
\vspace{-0.1pt}
\begin{description}
\item
\begin{tabular}{p{20 mm}p{145 mm}} 
$id$ & The id of the barrier. \\
$num\_procs$ & The number of threads or agents the barrier will block before
  continuing.
\end{tabular}
\end{description}

\noindent
{\bf Description}\\
This function is used to initialize Mobile-C Barrier variables for usage
by the \texttt{mc\_Barrier()} function. 
    \\

\noindent
{\bf Example}\\
Please see the example located at the directory
\texttt{ mobilec/demos/mc\_barrier\_example/ }. \\

\noindent

\noindent
{\bf See Also}\\
mc\_Barrier(), mc\_BarrierDelete(). \\

%\CPlot::\DataThreeD(), \CPlot::\DataFile(), \CPlot::\Plotting(), \plotxy().\\
