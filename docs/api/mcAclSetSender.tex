\rhead{\bf mc\_AclSetSender()}
\noindent
\vspace{5pt}
\rule{6.5in}{0.015in}
\noindent
\phantomsection
{\LARGE \bf mc\_AclSetSender()\index{mc\_AclSetSender()}}\\
\addcontentsline{toc}{section}{mc\_AclSetSender()}
\label{api:mc_AclSetSender()}

\noindent
{\bf Synopsis}\\
{\bf \#include $<$libmc.h$>$}\\
{\bf int mc\_AclSetSender}({\bf fipa\_acl\_message\_t*} acl, {\bf const char*} name, {\bf const char*} address );\\

\noindent
{\bf Purpose}\\
Set the sender on an ACL message.\\

\noindent
{\bf Return Value}\\
Returns 0 on success or non-zero on failure.

\noindent
{\bf Parameters}
\vspace{-0.1in}
\begin{description}
\item
\begin{tabular}{p{10 mm}p{145 mm}} 
$acl$ & An initialized ACL message. \\
$name$ & Sets the name of the sender. \\
$address$ & Sets the address of the sender. 
\end{tabular}
\end{description}

\noindent
{\bf Description}\\
This function is used to allocate and set the ``sender'' field of
an ACL message. If this function is called more than once on an ACL 
message, the original data in the ``sender'' field is overwritten.\\

\noindent
{\bf Example}\\
\noindent
{\footnotesize\verbatiminput{../demos/FIPA_compliant_ACL_messages/fipa_test/test2.xml}}

\noindent
{\bf See Also}\\
\texttt{
  mc\_AclSetPerformative(), mc\_AclAddReceiver(), mc\_AclAddReplyTo(), 
    \linebreak mc\_AclSetContent()
}

%\CPlot::\DataThreeD(), \CPlot::\DataFile(), \CPlot::\Plotting(), \plotxy().\\
