\rhead{\bf MC\_InitializeAgencyOptions()}
\noindent
\vspace{5pt}
\rule{6.5in}{0.015in}
\noindent
\phantomsection
{\LARGE \bf MC\_InitializeAgencyOptions()\index{MC\_InitializeAgencyOptions()}}\\
\addcontentsline{toc}{section}{MC\_InitializeAgencyOptions()}
\label{api:MC_InitializeAgencyOptions()}

\noindent
{\bf Synopsis}\\
{\bf \#include $<$libmc.h$>$}\\
{\bf int MC\_InitializeAgencyOptions}({\bf struct MCAgencyOptions\_s* } options);\\

\noindent
{\bf Purpose}\\
Initialize the agency options structure to default values. \\

\noindent
{\bf Return Value}\\
The function returns 0 on success and non-zero otherwise.\\

\noindent
{\bf Parameters}
\vspace{-0.1in}
\begin{description}
\item
\begin{tabular}{p{10 mm}p{145 mm}} 
$options$ & An uninitialized reference to a \texttt{struct MCAgencyOptions\_s} type variable.
\end{tabular}
\end{description}

\noindent
{\bf Description}\\
This function fills the agency options struct with default values. This
function will overwrite any values that have already been set in the
struct. \\

\noindent
{\bf Example}\\
\noindent
{\footnotesize\verbatiminput{../demos/miscellaneous/mc_sample_app/mc_sample_app.c}}

\noindent
{\bf See Also}\\
MC\_Initialize().

%\CPlot::\DataThreeD(), \CPlot::\DataFile(), \CPlot::\Plotting(), \plotxy().\\
