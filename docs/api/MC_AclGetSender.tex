\rhead{\bf MC\_AclGetSender()}
\noindent
\vspace{5pt}
\rule{6.5in}{0.015in}
\noindent
\phantomsection
{\LARGE \bf MC\_AclGetSender()\index{MC\_AclGetSender()}}\\
\addcontentsline{toc}{section}{MC\_AclGetSender()}
\label{api:MC_Acl_GetSender()}

\noindent
{\bf Synopsis}\\
{\bf \#include $<$libmc.h$>$}\\
{\bf int MC\_AclGetSender}({\bf fipa\_acl\_message\_t*} acl, {\bf char**} name, {\bf char**} address );\\

\noindent
{\bf Purpose}\\
Get the sender from an ACL message.\\

\noindent
{\bf Return Value}\\
Returns 0 on success or non-zero on failure.

\noindent
{\bf Parameters}
\vspace{-0.1in}
\begin{description}
\item
\begin{tabular}{p{10 mm}p{145 mm}} 
$acl$ & An initialized ACL message. \\
$name$ & (Output) Gets the name of the sender. \\
$address$ & (Output) Gets the address of the sender. 
\end{tabular}
\end{description}

\noindent
{\bf Description}\\
This function takes pointers to characters, automatically allocates space for
character strings, and makes copies of the names and addresses of an ACL message
onto those strings. The variables passed into the \texttt{name} and \texttt{address}
parameters of the function should be freed manually by the caller.

\noindent
{\bf Example}\\
\noindent
{\footnotesize\verbatiminput{../demos/FIPA_compliant_ACL_messages/fipa_test/test2.xml}}

\noindent
{\bf See Also}\\
\texttt{
  MC\_AclSetPerformative(), MC\_AclAddReceiver(), MC\_AclAddReplyTo(), 
    \linebreak MC\_AclSetContent()
}

%\CPlot::\DataThreeD(), \CPlot::\DataFile(), \CPlot::\Plotting(), \plotxy().\\
