\rhead{\bf mc\_ResumeAgency()}
\noindent
\vspace{5pt}
\rule{6.5in}{0.015in}
\noindent
\phantomsection
{\LARGE \bf mc\_ResumeAgency()\index{mc\_ResumeAgency()}}\\
\addcontentsline{toc}{section}{mc\_ResumeAgency()}
\label{api:mc_ResumeAgency()}

\noindent
{\bf Synopsis}\\
{\bf \#include $<$libmc.h$>$}\\
{\bf int mc\_ResumeAgency}({void});\\

\noindent
{\bf Purpose}\\
This function resumes the execution of an agency. \\

\noindent
{\bf Return Value}\\
The function returns 0 on success and non-zero otherwise.\\

\noindent
{\bf Parameters}
None. \\

\noindent
{\bf Description}\\
This function resumes the operation of the core threads of the Mobile-C
agency, such as the ACC, AMS, etc., after they have been halted by the 
\texttt{mc\_HaltAgency()} function.\\

\noindent
{\bf Example}\\
\noindent
% FIXME: Need an example here
%{\footnotesize\verbatiminput{../demos/multiple_agency_example/server.c}}

\noindent
{\bf See Also}\\
mc\_HaltAgency().

%\CPlot::\DataThreeD(), \CPlot::\DataFile(), \CPlot::\Plotting(), \plotxy().\\
