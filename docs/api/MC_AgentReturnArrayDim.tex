\rhead{\bf MC\_AgentReturnArrayDim()}
\noindent
\vspace{5pt}
\rule{6.5in}{0.015in}
\noindent
\phantomsection
{\LARGE \bf MC\_AgentReturnArrayDim()\index{MC\_AgentReturnArrayDim()}}\\
\addcontentsline{toc}{section}{MC\_AgentReturnArrayDim()}
\label{api:MC_AgentReturnArrayDim()}

\noindent
{\bf Synopsis}\\
{\bf \#include $<$libmc.h$>$}\\
{\bf int MC\_AgentReturnArrayDim}({\bf MCAgent\_t} agent, {\bf int} task\_num);\\

\noindent
{\bf Purpose}\\
Get the dimension of an array contained within a return agent.\\

\noindent
{\bf Return Value}\\
Returns the dimension of the array or -1 on failure.\\

\noindent
{\bf Parameters}
\begin{itemize}
\item \texttt{agent} : A return agent.
\item \texttt{task\_num} : This variable chooses which task within an agent to
retrieve the array dimension.
\end{itemize}


\noindent
{\bf Description}\\
This function finds the array dimension of an array contained within a return
agent. An agent may have multiple tasks, each with its own return value. The
\texttt{task\_num} argument chooses which task within an agent to obtain
information about.

\noindent
{\bf Example}\\
\noindent
{\footnotesize\verbatiminput{../demos/composing_agents/multi_task_example/client.c}}

\noindent
{\bf See Also}\\
\texttt{
  MC\_AgentReturnArrayExtent(), MC\_AgentReturnArrayNum()
}

%\CPlot::\DataThreeD(), \CPlot::\DataFile(), \CPlot::\Plotting(), \plotxy().\\
