\rhead{\bf MC\_SendAgentMigrationMessageFile()} 
\noindent
\vspace{5pt}
\rule{6.5in}{.01in}
\noindent
{\LARGE \bf MC\_SendAgentMigrationMessageFile()\index{MC\_SendAgentMigrationMessageFile()}} [Deprecated]\\ 
\phantomsection
\addcontentsline{toc}{section}{MC\_SendAgentMigrationMessageFile()}
\label{api:MC_SendAgentMigrationMessageFile()}

\noindent
{\bf Synopsis}\\
{\bf \#include $<$libmc.h$>$}\\
{\bf int MC\_SendAgentMigrationMessageFile}({\bf MCAgency\_t} $agency$, {\bf char *}$filename$, {\bf char *}$hostname$, {\bf int} $port$);\\

\noindent
{\bf Purpose}\\
Send an ACL mobile agent message saved as a file to a remote agency.

Please note that this function is deprecated. Please use the
\texttt{MC\_SendAgentFile()} function instead.\\

\noindent
{\bf Return Value}\\
The function returns 0 on success and non-zero otherwise.\\

\noindent
{\bf Parameters}
\vspace{-0.1in}
\begin{description}
\item
\begin{tabular}{p{20 mm}p{135 mm}}
$agency$ & A handle associated with an agency from which to send the ACL 
mobile agent message. A NULL pointer can be used to send the ACL message 
from an unspecified agency.\\ 
$filename$ & The ACL mobile agent message file to be sent.\\
$hostname$ & The hostname of the remote agency. It can be in number-dot 
format or hostname format, i.e., 169.237.104.199 or machine.ucdavis.edu.\\
$port$ & The port number on which the remote agency is listening. 
\end{tabular}
\end{description}

\noindent
{\bf Description}\\
This function is used to send an XML based ACL mobile agent message, which 
is saved as a file, to a remote agency. 
This function can be used without a running local agency.\\

\noindent
{\bf Example}\\
\noindent
{\footnotesize\verbatiminput{../demos/getting_started/hello_world/client.c}}

\noindent
{\bf See Also}\\

%\CPlot::\DataThreeD(), \CPlot::\DataFile(), \CPlot::\Plotting(), \plotxy().\\
