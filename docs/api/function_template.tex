\newpage
\rhead{\function}
\noindent
\vspace{5pt}
\rule{6.5in}{.01in}
\noindent
{\Large \function}\index{function}\\
\addcontentsline{toc}{section}{function}

\noindent
{\bf Synopsis}\\
A synopsis of the function syntax and a listing of relevant header files.
The function name is in bold face.\\
The parameters are in italic.\\

\noindent
[{\bf Syntax}] (optional)\\
A listing of the function syntax if the function has a variable number
of arguments.
The function name is in bold face.
The parameters are in italic.\\

\noindent
{\bf Purpose}\\
A brief description of the purpose of the function.\\

\noindent
{\bf Return Value}\\
The return value of function.\\

\noindent
{\bf Parameters}
\vspace*{-7.5pt}
\begin{description}
\item[{\it parameter\_name}] Description of the parameter.
\item[{\it parameter\_name}] Description of the parameter.
\end{description}

\noindent
{\bf Description}\\
A more detailed description of the function. \\

\noindent
[{\bf Algorithm}] (optional)\\
Optional information
regarding numerical algorithms. \\

\noindent
[{\bf Limitations}] (optional)\\
Limitations of the function.\\

\noindent
[{\bf Portability}] (optional)\\
Platform dependent information.
Platforms on which the functions and classes are not supported.\\

\noindent
[{\bf Example}] (optional)
{\footnotesize\verbatiminput{template/example/template}}
%[{\bf Example1}]
% For functions with multiple examples, each can be listed and numbered.

\noindent
[{\bf Output}] (optional)\\
Output of the example code.\\

\noindent
[{\bf See Also}] (optional)\\
A list of related classes and functions.\\

\noindent
[{\bf References}] (optional)\\
References for algorithms or methods used in the function.\\

