\rhead{\bf MC\_Steer()}
\noindent
\vspace{5pt}
\rule{6.5in}{0.015in}
\noindent
\phantomsection
{\LARGE \bf MC\_Steer()\index{MC\_Steer()}}\\
\addcontentsline{toc}{section}{MC\_Steer()}

\noindent
{\bf Synopsis}\\
{\bf \#include $<$libmc.h$>$}\\
{\bf int MC\_Steer}({\bf MCAgency\_t} attr, {\bf int} (*funcptr)({\bf void*} data), {\bf void*} arg);\\

\noindent
{\bf Purpose}\\
The MC\_Steer function initialized and runs a function containing an algorithm. 
The function enables the steering functionality of the algorithm so that it may accept 
command during runtime to change the execution of the algorithm. 
For more information, please see the example and the demo located in the 
demos/steer\_example/ directory.\\ 

\noindent
{\bf Return Value}\\
The function returns 0 on success, or a non-zero error code on failure. \\

\noindent
{\bf Description}\\
The {\bf MC\_Steer} function is designed execute an algorithm in a fashion which enables
that algorithm to be steered or modified on-the-fly during runtime. See the demo and
the example for more details. \\

\noindent
{\bf Example}\\
\noindent
{\footnotesize \verbatiminput{../demos/miscellaneous/steer_example/server.c}}
%Compare with output for examples in \CPlot::\Arrow(), \CPlot::\AutoScale(),
%\CPlot::\DisplayTime(), \CPlot::\Label(), \CPlot::\TicsLabel(), 
%\CPlot::\Margins(), \CPlot::\BoundingBoxOffsets(), \CPlot::\TicsDirection(),\linebreak
%\CPlot::\TicsFormat(), and \CPlot::\Title().
%{\footnotesize\verbatiminput{template/example/Data2D.ch}}

\noindent
{\bf See Also}\\
MC\_SteerControl()

%\CPlot::\DataThreeD(), \CPlot::\DataFile(), \CPlot::\Plotting(), \plotxy().\\
