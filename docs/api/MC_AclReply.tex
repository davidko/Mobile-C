\rhead{\bf MC\_AclReply()}
\noindent
\vspace{5pt}
\rule{6.5in}{0.015in}
\noindent
\phantomsection
{\LARGE \bf MC\_AclReply()\index{MC\_AclNew()}}\\
\addcontentsline{toc}{section}{MC\_AclReply()}
\label{api:MC_AclReply()}

\noindent
{\bf Synopsis}\\
{\bf \#include $<$libmc.h$>$}\\
{\bf int MC\_AclReply}({\bf fipa\_acl\_message\_t*} acl\_message);\\

\noindent
{\bf Purpose}\\
Automatically generate an ACL message addressed to the sender of an
incoming ACL message..\\

\noindent
{\bf Return Value}\\
A newly allocated ACL message with the 'receiver' field initialized, or
NULL on failure.\\

\noindent
{\bf Parameters}
\vspace{-0.1in}
\begin{description}
\item
\begin{tabular}{p{10 mm}p{145 mm}} 
$acl\_message$ & The message to generate a reply to.
\end{tabular}
\end{description}

\noindent
{\bf Description}\\
This function is designed to make replying to received ACL messages easier.
The function automatically generates a new ACL message with the correct
destination address to reach the sender of the original message.\\

\noindent
{\bf Example}\\
\noindent

\noindent
{\bf See Also}\\
\texttt{
  MC\_AclNew(), MC\_AclPost(), MC\_AclRetrieve(), MC\_AclSend(), \linebreak 
    MC\_AclWaitRetrieve()
}

%\CPlot::\DataThreeD(), \CPlot::\DataFile(), \CPlot::\Plotting(), \plotxy().\\
