\rhead{\bf MC\_SetThreadOff()}
\noindent
\vspace{5pt}
\rule{6.5in}{0.015in}
\noindent
{\LARGE \bf MC\_SetThreadOff()\index{MC\_SetThreadOff()}}\\
\phantomsection
\addcontentsline{toc}{section}{MC\_SetThreadOff()}

\noindent
{\bf Synopsis}\\
{\bf \#include $<$libmc.h$>$}\\
{\bf int MC\_SetThreadOff}({\bf MCAgencyOptions\_t} $*options$, {\bf enum threadIndex\_e} $thread$);\\

\noindent
{\bf Purpose}\\
Set a particular thread to not execute upon Mobile-C initialization.\\

\noindent
{\bf Return Value}\\
This function returns 0 on success and non-zero otherwise. \\

\noindent
{\bf Parameters}
\vspace{-0.1in}
\begin{description}
\item               
\begin{tabular}{p{10 mm}p{145 mm}}
$options$ & An allocated MCAgencyOptions\_t variable.\\
$thread$ & A thread index.
\end{tabular}
\end{description}

\noindent
{\bf Description}\\
This function is used to modify the Mobile-C startup options. 
It is used to disable threads that may otherwise be enabled. 
The threads which may be modified are
\vspace{-0.1in}
\begin{description}
\item               
\begin{tabular}{p{40 mm}p{125 mm}}
MC\_THREAD\_AI : & Agent Initializing Thread - Create agent from incoming messages.\\
MC\_THREAD\_AM : & Agent Managing Thread - Manage active agents.\\
MC\_THREAD\_CL : & Connection Listening Thread - Listen incoming connections.\\
MC\_THREAD\_MR : & Message Receiving Thread - Handle incoming connections and recieve agent messages.\\
MC\_THREAD\_MS : & Message Sending Thread - Handle outgoing connections and send agent messages.\\
MC\_THREAD\_CP : & Command Prompt Thread - Handle an interactive user command prompt.
\end{tabular}
\end{description}

\noindent
{\bf Example}\\
\begin{verbatim}
MCAgencyOptions_t options;
MCAgency_t agency;

/* Turn the listen thread off. We will receive our messages 
   in another method. */
MC_SetThreadOff(&options, MC_THREAD_AI);

/* Start the agency with no listen thread*/
agency = MC_Initialize(5050, &options);

/* etc ... */
\end{verbatim}

\noindent
{\bf See Also}\\
MC\_SetThreadOn()\\
%\CPlot::\DataThreeD(), \CPlot::\DataFile(), \CPlot::\Plotting(), \plotxy().\\

