\rhead{\bf MC\_ResetSignal()}
\noindent
\vspace{5pt}
\rule{6.5in}{0.015in}
\noindent
{\LARGE \bf MC\_ResetSignal()\index{MC\_ResetSignal()}}\\
\phantomsection
\addcontentsline{toc}{section}{MC\_ResetSignal()}

\noindent
{\bf Synopsis}\\
{\bf \#include $<$libmc.h$>$}\\
{\bf int MC\_ResetSignal}({\bf MCAgency\_t} $agency$);\\

\noindent
{\bf Purpose}\\
This function is used to reset the Mobile-C signalling system. 
It is intended to be used after returning from a call to function 
{\bf MC\_WaitSignal()}.\\

\noindent
{\bf Return Value}\\
This function returns 0 on success and non-zero otherwise.\\

\noindent
{\bf Parameters}
\vspace{-0.1in}
\begin{description}
\item               
\begin{tabular}{p{10 mm}p{145 mm}}
$agency$ & A handle to a running agency.
\end{tabular}
\end{description}

\noindent
{\bf Description}\\
This function is used to reset the Mobile-C signalling system. 
System signals are triggered by certain events in the Mobile-C library. 
This includes events such as the arrival of a new message or mobile agent, and 
the departure of a mobile agent, etc. 
If function {\bf MC\_WaitSignal()} is used to listen for one of these events, 
function {\bf MC\_ResetSignal()} must be called in order to allow Mobile-C to 
resume with it's operations.\\

\noindent
{\bf Example}\\
\noindent
{\footnotesize\verbatiminput{../demos/composing_agents/multi_task_example/client.c}}

\noindent
{\bf See Also}\\
MC\_WaitSignal() \\

%\CPlot::\DataThreeD(), \CPlot::\DataFile(), \CPlot::\Plotting(), \plotxy().\\
