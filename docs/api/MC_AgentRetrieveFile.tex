\rhead{\bf MC\_AgentRetrieveFile()}
\noindent
\vspace{5pt}
\rule{6.5in}{0.015in}
\noindent
\phantomsection
{\LARGE \bf MC\_AgentRetrieveFile()\index{MC\_AgentRetrieveFile()}}\\
\addcontentsline{toc}{section}{MC\_AgentRetrieveFile()}

\noindent
{\bf Synopsis}\\
{\bf \#include $<$libmc.h$>$}\\
{\bf int MC\_AgentRetrieveFile}({\bf MCAgent\_t} $agent$, 
                                  {\bf int} $tasknum$,
                                  {\bf const char*} $name$,
                                  {\bf const char*} $filepath$,
																	);\\

\noindent
{\bf Purpose}\\
This function is used to retrieve and save a file to from agent.\\

\noindent
{\bf Return Value}\\
The function returns 0 on success or a non-zero error code on failure.\\

\noindent
{\bf Parameters}
\vspace{-0.1in}
\begin{description}
\item
\begin{tabular}{p{30 mm}p{125 mm}} 
$agent$ & A fully initialized agent handle.\\
$tasknum$ & The task in which to retrieve the file. \\
$name$ & An alias to identify the attached file.\\
$filepath$ & The path to save the file. Local paths are calculated from the
execution directory of the agency.\\
\end{tabular}
\end{description}

\noindent
{\bf Description}\\
This function is used to retrieve a file from an agent task. The file must
be attached to the agent from a prior call to \texttt{MC\_AgentAttachFile()}. 
The executing agency must have write permissions to save the file to the
correct location.

\noindent
{\bf Example}\\
\noindent
Please see the example code attached with the documentation for \texttt{MC\_AgentAttachFile}.

\noindent
{\bf See Also}\\
\texttt{MC\_AgentAttachFile(), MC\_AgentListFiles()}

%\CPlot::\DataThreeD(), \CPlot::\DataFile(), \CPlot::\Plotting(), \plotxy().\\
